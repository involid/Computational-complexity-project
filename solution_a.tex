\section*{Решение (а)}

Пусть $f(G)$ - наименьший $k$ разрез графа $G$.
Для доказательства NP-трудности необходимо доказать, что $NP \subset P^f$. 
Докажем пренодлежность NP-полной задачи проверки существования в графе \( G \) клики размера \( k \) к $P^f$.

Пусть дан граф \( G(V, E) \), число \( k \).
Пусть \( |V| = n \).
Построим граф \( G'(V', E') \) и определим веса для \( E' \), то есть функцию \( w : E \rightarrow \mathbb{R}_+ \).
Для построения \( G' \) добавим в \( G \) две вершины \( v_{top} \) и \( v_{bottom} \) и соединим их со всеми остальными вершинами.
Формально, \( V' = V \cup \{v_{top}\} \), \( E' = E \cup \{(v_{top}, v) : v \in V\} \cup \{(v_{top}, v_{bottom})\} \).
Веса ребер из \( E \) будут равны 1, ребра из \( v_{top} \) в \( v \in V \) будут иметь вес \( n^2 - \deg(v) \) (\( deg(v) \) — степень вершины \( v \) в графе \( G \)).
Таким образом, построение \( (G', w') \) осуществимо за полиномиальное время.
Утверждается, что существование клики размера \( k \) в графе \( G \) эквивалентно условию, что minimum \( k \)-cut \( (G', w', k + 1) \) равен \( kn^2 - C_k^2 \), что и будет доказательством сводимости.

Для доказательства утверждения покажем, что из существования клики размера \( k \) в графе \( G \) следует, что minimum \( k + 1 \)-cut \( (G', w', k + 1) \) не превышает \( kn^2 - C_k^2 \), и наоборот, если minimum \( k + 1 \)-cut \( (G', w', k) \) не превышает \( kn^2 - C_k^2\), то в графе \( G \) существует клика размера \( k\) и в $G'$ minimum \( k + 1 \)-cut \( (G', w', k + 1) \) в точности равняется \( kn^2 - C_k^2 \).

Пусть в $G$ есть клика $V'$ размера $k$. 
Посчитаем сумму весов ребер, которые нужно убрать, чтобы вершины из $V'$ были изолированы, в результате чего получим $k+1$ компоненту связности.
Веса ребер вершин $V'$ и $v_{top}$ равны $k n^2 - \sum_{v \in V'} deg(v)$.
$\sum_{v \in V'} deg(v)$ в $G$ — это количество ребер из $V'$ плюс удвоенное количество ребер в $V'$.
Количество ребер в $V'$ равно $C_k^2$, значит количество всех ребер хотя бы с одним концом из $V'$ в $G$ будет $\sum_{v \in V'} deg(v) - C_{k}^2$.
Веса всех этих ребер равны 1.
Суммарный вес разреза равен $kn^2 - C_k^2$.

Теперь пусть существует $k+1$ разрез, у которого вес не более $kn^2 - C_k^2$.
Вершины $G'$ отличаются от ввершин $G$ добавлением одной вершины $v_{top}$.
Значит, есть $k$ компонент связности с некоторыми вершинами только из $G$.
Пусть $V'$ — вершины этих компонент.
Предположим, что в какой-то из этих компонент связности есть больше одного ребра.
Тогда $|V'| \geq k+1$.
В таком случае все ребра, соединяющие $v_{top}$ с какой-то из этих $V'$ вершин, должны присутствовать в разрезе.
Сумма весов таких ребер $\sum_{v \in V'} (n^2 - deg(v)) \geq (k+1)n^2 - \sum_{v \in V}deg(v) \geq (k+1)n^2 - n(n-1) = kn^2 + n$, что уже больше веса разреза.
Значит $|V'| = k$ и в каждой из компонент, кроме одной, 1 вершина.
Тогда разрез — это вершины с концом из $V'$.
Похожий вес разреза мы уже считали при условии, что $V'$-клика.
Сумма весов ребер из $v_{top}$ в $V'$ есть $kn^2 - \sum_{v \in V'}deg(v)$.
Сумма весов ребер из $V'$ в $V$ есть $\sum_{v \in V'} deg(v) - |\{(v_1, v_2) \in E : v_1, v_2 \in V'\}|$, так как в сумме степеней ребра из $V'$ учитываются дважды, причем количество ребер из $V'$ не больше $C_k^2$ и равно $C_k^2$ только когда $V'$-клика.
Суммарный вес получается $kn^2 - |\{(v_1, v_2) \in E : v_1, v_2 \in V'\}| \geq kn^2 - C_k^2$, причем равенство достигается, когда $V'$ клика.
Получили, что вес разреза в точности равен $kn^2 - C_k^2$ и в $G$ есть клика размера $k$.

Таким образом одним обращением к оракулу $f$, можно за полиномиальное время решить задачу проверки наличия клики размера $k$ в графе $G$.